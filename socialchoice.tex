\chapter{Social choice}
\label{ch:socialchoice}

\section{Alternative views on ethics}
Social science is about three questions: what if; so what; and what to do.\footnote{Natural science is only about what if questions. The incomplete education of natural scientists leads to endless confusion and discussion when they leave the ivory tower to partake in policy debates.} What would happen to eutrophication if fertilizer is taxed based on its nitrogen content? Do the costs of restricting nitrogen application to farmers and consumers matter more than the impacts of eutrophication on nature and recreation? Is it better to tax fertilizer or to forbid certain applications?

The first question, what if, is positive. It requires an as-accurate-as-possible, as-objective-as-feasible description of the relevant parts of the world as it is. The third question, what to do, is normative. You cannot rank options without a clear idea of what is better and worse. The second question is a mix of normative and positive\textemdash the measurement of costs and benefits is positive, but the decision what to include and exclude is normative.

The idea of what is better and worse is key to answering normative questions. Normative questions are rife in public policy advice and evaluation, including all aspects of environmental policy.

Environmental economics, when informing public policy, has a more profound normative problem than other branches of applied public economics. The economic analysis of education, health care or labour markets is invariably confronted with making trade-offs between people. In environmental economics, we often make trade-offs between people too, but also between humans and non-humans.

It is therefore important to discuss what we mean by better and worse. Economics is by and large based on a utilitarian ethics. Alternative views are rarely discussed. Yet, utilitarianism is a minority view among moral philosophers.

I discuss three major strands in moral philosophy, naturalist ethics and two strands of humanist ethics, libertarianism and utilitarianism. I particularly focus on utilitarianism, because economics is based on that. There are other, less relevant strands of ethics, but these are not discussed here.

\section{Universality}
Immanuel Kant is one of the most important philosophers in history. Although he died more than two centuries ago, young philosophers still study his work, only to discover what a singularly obscure writer he was. Among other innovations, Kant introduced the \emph{moral agent} and the \emph{universal law}.

A moral agent is the unit of analysis in ethics. Rights and duties can be bestowed on moral agents and on moral agents only.

Kant's universal law holds that, if a rule applies to one moral agent, it applies to all moral agents. For instance, assume that you and I are both moral agents. If I argue that you cannot chop off my arm, then I cannot chop off your arm either.

\section{Naturalism}
Naturalist moral philosophy is centred on the question who is a moral agent. This is perhaps best illustrated with a review of political rights.

It used to be that only the views of rich white male Protestants mattered. Rich white male Protestants argued that only they mattered because only they were worthy and capable.

Then along came Jeremy Bentham, a rich white male Protestant, who wrote that government should strive for the greatest good for the greatest number. This was a radical proposition. Poor lives matter. The government should serve not just the elite, but the less fortunate too. Bentham had no truck with Catholics, non-whites or women, but he did argue in favour of people less wealthy than he was.

This intervention set off a cascade. If you cannot argue that someone else is worthy only if she is just like you, then you need a different criterion to delineate moral agents from other entities.

Catholics were emancipated first. You cannot maintain that someone is less worthy because he reads a different translation of the same book. Women were next. The notion that women are too hysterical to own property or vote was shown to be a self-serving lie than men like to tell each other. Skin colour was last. A physiological adaptation to the intensity of sunlight has nothing to do with other human capabilities.

However, if as a rich white male Protestant you cannot argue that only rich white male Protestant matter, then as a human you cannot argue that only humans matter. It is a self-serving argument.

Some people argue that the right delineation is a sense of self. If you put a parakeet in front of a mirror, it will sing to its mirror image\textemdash and will continue to do so even if the mirror does not answer back. A parakeet does not have a concept of self and no concept of other. A moral rule that constrains the self as it treats others is therefore meaningless to a parakeet. That said, other primates, dolphins and elephants do recognize themselves in the mirror. Some human rights would therefore also apply to other higher animals. In 2015, a judge in Argentina \href{https://edition.cnn.com/2014/12/23/world/americas/feat-orangutan-rights-ruling/}{ruled} that, since Sandra had broken no laws, it was illegal to hold her captive. Sandra is an orangutan.

Utilitarianism is about minimizing pain and maximising pleasure. If that is the ethical starting point, then what sets a moral agent apart is not her ability to self-identify, but rather her ability to experience pain and pleasure. Cats may not recognize themselves in the mirror\textemdash there are countless videos on YouTube that prove this point\footnote{Here is an \href{https://twitter.com/Pandamoanimum/status/1283753313134149633}{exception}.}\textemdash but cats sure are able to experience pain and pleasure. Many people oppose torture of animals as they oppose torture of humans, and many countries have laws against cruelty to animals.

The human ability to experience pain and pleasure derives from our central nervous system, with nerves throughout the body and a brain to coordinate it all. Worms do not have a central nervous system. If you chop a human in two, you have two halves of a dead human. If you chop a worm in two, you have two worms. But if you cannot argue that someone is unworthy because they have a different skin tone\textemdash a physical characteristic\textemdash you cannot argue that something is unworthy because it lacks a central nervous system. Octopuses have nine brains. They are smart and have complex personalities. Extending human rights to all animals may seem extreme to people who grew up in a society influenced by Christianity or Islam, but it is a common position in Hinduism.

Trees too signal distress in a way that is alien to humans but clearly recognizable nonetheless. Fruitarians extend the well-accepted rule against cannibalism to all living beings.

The cascade does not stop there. The occasional media flurry on the discovery of life on Mars is rooted in our inability to distinguish organic material from dead material. At the macroscopic scale, it is easy to tell a bear from a tree from a rock. At the microscopic scale, such distinctions are blurred. Viruses, for instance, are somewhere between alive and not. Deep ecologists like Aldo Leopold and Arne Naess argue that non-living entities indeed have a right to integrity of body just like humans do. Certain religions recognize abiotic entities, such as rivers or mountains, as spiritual beings worthy of respect and protection.

\section{Libertarianism}
Libertarianism is one of the schools of humanist moral philosophy. It grants rights and duties to humans only, although some of the reasoning can readily be extended to other species.

As the name suggests, libertarianism is about individual rights and liberties. John Locke argued property is just if it is acquired through labour. That is, if someone goes into the forest, cuts down the trees and starts cultivating the land, then that land is theirs. When Locke wrote this, King James VII and II was trying to establish an absolute monarchy in England and Scotland, including the notion that all land belongs to the crown. Locke disagreed. This thinking is reflected in the US Homestead Act of 1862, and it reflects Germanic traditions of property law.

Locke's idea of just property is impractical. Robert Nozick added that property is just if acquired through labour or obtained through free consent. Just property remains just after voluntary exchange.

Libertarianism is thus only concerned with procedural justice. What matters is how you get there, not where you end up. An unequal distribution of resources is of no concern to a libertarian provided that the rich got rich by legal means.

The role of the state is rather limited in libertarianism. The government should guard against unjust holdings, such as theft. The government should also guard against negative externalities, which are involuntary impositions on the liberties and properties of others. That is all.

Libertarians argue that taxation is theft. Governments can therefore not distribute resources from the rich to the poor. The government may provide public goods, but contributions to that should be strictly voluntary.

\section{Utilitarianism}
Utilitarianism is the polar opposite of libertarianism. Utilitarianism is consequential justice. What matters is where you end up, not how you got there.

At the individual level, utilitarianism is about pain and pleasure. At the social level, utilitarianism is about the greatest good for the greatest number. In narrow definitions of utilitarianism, this means the sum total of the utility of people. In broad definitions of utilitarianism, this means some aggregate of the utility of people and perhaps animals.\footnote{Some philosophers argue that broad utilitarianism is not utilitarianism at all, but I have yet to see a cogent argument why not.}

The government should deliver the greatest good for the greatest number. It does not matter how. An autocratic government that brings material welfare to its citizens is better, according to utilitarians, than a democratic government of a poor country.

Broad interpretations of utilitarianism can be captured with a Bergson-Samuelson-Atkinson welfare function
\begin{equation}
\label{eq:Bergson}
    W = W(U_1, U_2, ..., U_n) = \frac{1}{1-\gamma} \sum_{i=1}^n U_i^{1-\gamma}
\end{equation}
where $W$ denotes social welfare and $U_i$ the utility of individual $i=1,2, ..., n$. The right-hand side is due to Anthony Atkinson, the middle part was independently suggested by Abram Bergson and Paul Samuelson. The parameter $\gamma$ is relative inequity aversion.

At the margin, individuals $i$ and $j$ contribute to social welfare as follows
\begin{equation}
    \frac{\frac{\partial W}{\partial U_i}}{\frac{\partial W}{\partial U_j}} = \left (\frac{U_j}{U_i} \right )^\gamma
\end{equation}
If $\gamma=0$, the social planner is inequity neutral. $W=\sum_{i=1}^n U_i$. It does not matter whether the utility of $i$ or $j$ goes up, because the ratio of their marginal contributions to welfare is always equal to one.

For $\gamma>0$, the social planner is inequity averse. If $i$ is happier than $j$, $U_i > U_j$, then $\left (\frac{U_j}{U_i} \right )^\gamma < 1$. That is, a utility gain for happy $i$ is less important than a utility gain for miserable $i$.

There is another way to see the same thing:
\begin{equation}
\label{eq:Atkinson}
    W = \begin{cases}
    \min_i U_i & \text{if } \gamma \uparrow \inf \text{ (Rawls)}\\
    \prod_i U_i & \text{if } \gamma=1 \text{ (Bernoulli-Nash)}\\
    \sum_i U_i & \text{if } \gamma=0 \text{ (Bentham)}\\
    \max_i U_i & \text{if } \gamma \downarrow -\inf \text{ (Nietzsche)}
\end{cases}
\end{equation}
As $\gamma$ grows, more and more emphasis is placed on the plight of the worst-off in society.

Arrow's Impossibility Theorem shows that a welfare function cannot be an aggregate of individual preferences. That does not mean that you should not use social welfare functions. It does mean that you should be aware of their limitations.

\section{Critiques of utilitarianism}
Utilitarianism is unpopular outside economics. Besides the alternative schools of moral philosophy, I highlight the work of two criticasters.

John Rawls is seen as one of the key ethicists of the 20th centry. Rawls thought and wrote about what a just society would look like. He argued that a just society would be one that everyone in that society would agree on, if they were free to decide, rational, and impartial. For impartiality, Rawls introduced his \emph{veil of ignorance}: You can be impartial only if you do not know what position you hold in society, if you do not know how skilled or talented you are, and if you do not know your attitudes towards risk, inequality and such things.

On that basis, Rawls argued that a just society would be as free as possible. Anyone should be who they like to be and do what they want to do, as long as it does not infringe on other people's liberties. Rawls also argued that a just society minimises resource difference. Incomes should only deviate if that income difference makes everyone better off, and if that income difference is attached to position. For example, doctors and firefighters have to be on standby for 24/7, so it stands to reason that they are compensated for their sacrifice lest no one wants to take that job. Doctors are trained for a longer period than nurses so it is reasonable to compensate them for that\textemdash but not for the fact they were born into a different class or have a greater aptitude for academic study. Rawls' just society is nothing like ours.

As shown in Equation (\ref{eq:Atkinson}, a Rawlsian income distribution can be captured in a social welfare function. Other aspects of Rawlsian justice, such as maximum freedom, cannot.

Amartya Sen is an economist and Nobel laureate. A utilitarian would applaud the government of Singapore which is fairly authoritarian but has made its people rich. The governments of South Korea and Taiwan were similarly down on freedom but up on economic growth, and were removed by its people. Not everyone agrees that material wealth is all that matters. Sen noted a deeper problem. An aggregate of individual utilities, cannot reflect properties of society, such as democratic freedoms: A Bergson-Samuelson function has that $W=W(U_1, U_2, ..., U_n)$, $W \neq W(U_1, U_2, ..., U_n, F)$, where $F$ stands for freedom. Nor is it easy to add freedom as an attribute to a utility function: Freedom is not consumed or produced in any conventional meaning of those words, and it is property of society rather than an individual.

Sen also highlighted altruism. At first sight, altruism is compatible with utilitarianism. If person $i$ cares about person $j$, $U_i=U(C_i, U_j)$. That is, besides on her own consumption $C_i$, utility is a function of the other person's happiness (or perhaps perceived happiness or consumption). There are two problems with this. First, the construction of the aggregate demand curve assumes that individual demand curves are independent of each other. The vertical aggregation of demand curves allows for the kind of altruism that affects our well-being but not our behaviour. You can work your way around this (not in an undergraduate class though) but then a different problem emerges. A social welfare function is meant to give policy advice. A social welfare function that includes altruism has a fundamental inequity build-in. If your utility depends on your well-being and the well-being of the people you care about, then a social welfare function that reflects your utility double-counts the utility of those you care about. That would not be a problem if care were equally distribution. If not, such a social welfare function is biased towards popular people and biased against unpopular ones.

Sen further wrote about agency. People behave differently in different roles. You are different around your family then when you are with friends or colleagues. This is first and foremost a positive problem, an issue with describing and predicting what economic agents do. It is hard to construct a utility function and budget constraints, that manifests itself in first-order condition that solve differently depending on who else is in the room. But this also affects the social evaluation of individual behaviour, which is after all what a welfare function is about. A middle-aged man chatting up a young woman in a bar is met with pity, a middle-aged male professor chatting up a female student after class stands accused of harassment and abuse of power. A business man giving preferential treatment to an old friend is giving away his own money, a politician gives away public funds. Different roles come with different expectations and responsibilities, so that a transaction can be legitimate and welfare-improving in one circumstance but not in another.

Utilitarianism is therefore an ethical system with many drawbacks. Applications of social welfare functions should always be treated with caution and inspected for flaws.