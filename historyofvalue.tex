\chapter{History of value}
\label{ch:histvalue}

\section{Proto-economics}
Adam Smith (\href{https://www.econlib.org/library/Smith/smWN.html?chapter_num=7#book-reader}{Wealth of Nations I-IV, 1776}) formulated the paradox of value thus:
\begin{quote}
    The word VALUE, it is to be observed, has two different meanings, and sometimes expresses the utility of some particular object, and sometimes the power of purchasing other goods which the possession of that object conveys. The one may be called 'value in use', the other 'value in exchange'. The things which have the greatest value in use have frequently little or no value in exchange; and on the contrary, those which have the greatest value in exchange have frequently little or no value in use. Nothing is more useful than water: but it will purchase scarce any thing; scarce any thing can be had in exchange for it. A diamond, on the contrary, has scarce any value in use; but a very great quantity of other goods may frequently be had in exchange for it.
\end{quote}
Smith's words echoed earlier writers. The oldest surviving account is by Plato (\href{http://classics.mit.edu/Plato/euthydemus.1b.txt}{Euthydemus, 304BC}), who, remarking on Crito's easily copied oratory style, wrote
\begin{quote}
for only what is rare is valuable; and ''water'' which [...] is the ''best of all things'' is also the cheapest.
\end{quote}

The Ancient Greek philosophers, led by Aristotle, had little interest in price formation and why prices \emph{are} as they are. They gave little consideration to value in exchange. Instead, they focused on value in use, using ethics to reason what values \emph{should} be. They worried about the deviation of the actual price from the right price. The market was deemed \emph{immoral}.

Some 750 years later, St Augustine replaced Aristotelian ethics by Christian theology. Like Aristotle, he was mainly interested in what values should be. Unlike Aristotle, he argued that the right price can only be found by examining the will of God. Working some 850 years after, Thomas Aquinas by and large adopted Augustine's position.

In 1662, William Petty wrote
\begin{quote}
    Labour is the Father and active principle of Wealth, as Lands are the Mother.
\end{quote}
Petty maintained the notion that there is an absolute yardstick for value, but he replaced an intangible God with tangible assets.

The 18th century Physiocrats, led by Fran\c{c}ois Quesney, followed Petty to a limited extent. They put God at one remove from value. Quesnay argued that society should be based on the \emph{ordre naturel}, the laws of nature as dictated by God. Agriculture was society's interface with nature, and therefore only agriculture can yield a net surplus. Other economic activities do not add value, take as much in inputs as they make in outputs. Note that, according to the Physiocrats, it is nature that creates value. Farmers merely reap that value. The source of value is the land.

The land theory of value thus replaced one absolute yardstick\textemdash God\textemdash with another\textemdash land. It is easy to see how the Physiocratic theory of value served the landed elite of France. Quesney was a land-owner himself, and served at the court of Louis XV.

\section{Classical economics}
Adam Smith disagreed with the Physiocrats, arguing that labour rather than land is the true source of value.\footnote{Ibn Khaldun had earlier developed a labour theory of value, but this was not known in Europe until much later.} David Ricardo and Karl Marx further elaborated Smith's labour theory of value.

Smith, Ricardo and Marx maintained the notion, going back to Aristotle, that value is absolute. Marxian economists held up another part of the Aristotelian tradition: If the market price deviates from the labour value, that is because the market is \emph{immoral}.

Smith and Ricardo instead argued that the market is \emph{moral}. The invisible hand guaranteed the greatest good for the greatest number.

\section{Neo-classical economics}
The neo-classical revolutionaries argued that value is relative. They abandoned the labour theory of value of Smith and Marx and the earlier absolute value theories, whether anchored on land or religion or morality.