\chapter{Sustainability}
\label{ch:sustainability}
Sustainability is a core concept in environmental policy. It is sometimes interpreted as a property of the social welfare function, but more commonly as a constraint we should impose on any decision we make.

\section{Roots}
John Stuart Mill \href{https://www.gutenberg.org/files/30107/30107-pdf.pdf}{(1848, Principles of Political Economy)} wrote
\begin{quote}
    If the earth must lose that great portion of its pleasantness which it owes to things that the unlimited increases of wealth and population would extirpate from it, for the mere purpose of enabling it to support a larger, but not a happier or better population, I sincerely hope, for the sake of posterity, that they will be content to be stationary long before necessity compels them to it.
\end{quote}
Like his fellow Classical economists, see Section \ref{sc:malthus}, Mill believed that economic growth must come to a halt. In the above passage, he argues that economic expansion should stop before economic output reaches its maximum because there is "pleasantness" apart from material wealth. He further argues that we should do this not for our own sake but for those who come after. Mill thus captured the essence of sustainability.

Gro Harlem Brundtland and colleagues \href{https://sustainabledevelopment.un.org/content/documents/5987our-common-future.pdf}{(1987, Our Common Future)} give the following definition
\begin{quote}
    Sustainable development is development that meets the needs of the present without compromising the ability of future generations to meet their own needs.
\end{quote}
This is intuitively appealing. We should develop but not hurt our children. The appeal of the Brundtland report led to a rapid and widespread support of sustainability. Every country and every organization wants to develop sustainably. Sustainability is the shared end goal, sustainabilizability the intermediate target.

The Brundtland definition is vague. What are needs? Food and water are beyond dispute, but I sometimes have a desperate urge to watch a silly show on TV. What is the ability to meet needs? Are we content if our descendants could have met their needs but did not? And what does compromising that ability mean? If their ability is reduced by 10\% is it compromised, or is the threshold at 90\%?

Brundtland's vagueness is her strength. The direction of travel is clear, but there is no precise prescription, which allows people from different convictions to subscribe to sustainable development.

Academics do not like vagueness. There have therefore been many attempts to give a precise definition of sustainability. I review these below, in three major groups.

\section{Weak sustainability}
Jack Pezzey argued that sustainability means that utility should not fall. We want to sustain human happiness. Utility is an elusive concept, though. John Hartwick therefore argued that sustainability means that consumption should not fall. In 1977, he formulated the Hartwick Rule: A constant level of consumption can be maintained perpetually from an environmental endowment if all the scarcity rents (net price, profit) from resource extraction are invested in other capital. That is, we should keep the principal intact, live off the service flow alone. Robert Solow argued that, with finite resources, non-declining consumption is indeed constant.

The Hartwick Rule appears sensible\textemdash do not run down your capital stock\textemdash and precise. Nor does the Hartwick Rule specify whose consumption should be kept constant, however. Is it the median consumer's? Everyone's? Nor does it say at what time-scale should consumption be constant? Every year, every decade, every century? And it does not define the bundle of consumption, however. Is it the same bundle over time?

In order to overcome the last problem, Talbot Page and Robert Solow suggested that production possibilities should not fall. This avoids the problem of defining consumption for now and forever. As an analysis of sustainability is done by current scholars, assumptions about future consumption necessarily reflect current preferences. A focus on production is less paternalistic. Problems of time scale and representation remain, however.

A key aspect of all these definitions is that substitution is allowed. The Hartwick Rule explicitly says that it is fine to run down natural resources as long as the revenue is used to build up physical capital. According to these \emph{weak} definitions of sustainability, what matters is that human welfare will not fall. A walk in the forest can be replaced with a walk in virtual reality.

\section{Strong sustainability}
The defining feature of weak sustainability is that human-made capital can substitute for natural capital. Advocates of strong sustainability disagree. Some go as far as arguing that sustainability means that natural capital stocks should be maintained.

This has far-reaching implications. Fossil fuels can no longer be used, or fossil water. Trees can only be cut for lumber if new trees are planted. Building materials would be scarce, a mining for sand and limestone comes to an end.

If all natural capital stocks are to stay at their current levels, human activity would be severely curtailed. A new road takes up space, necessarily taking away space from something else. Building a new house thus means taking down an old one. If strong sustainability is defined at the local level, the old house needs to be in the same county.

In practice, therefore, some substitution must be allowed. But at what spatial scale? If forest has to make way for a new road in England, should new trees be planted in England or is it fine to plant new trees in Hungary? To what extent do new trees substitute for old trees, which provide a much richer habitat for birds, plants and insects?

What natural capital stocks should be maintained? Ecosystems, species, or genes? The polar bear is a subspecies of the brown bear. Not much genetic diversity would be lost if the polar bear goes extinct. Or maybe we do want to preserve individual species from extinction, whether caused by humans or evolutionary dynamics. And what should we do with viruses and pests? Should we let malaria roam freely?

The desire to maintain natural capital stocks reflects a static view of the natural environment. Things should stay as they are or return to how they used to be. An alternative view on strong sustainability argues that, instead, services from natural capital stocks should not decline. What matters is that there are enough photosynthesizing plants to make sufficient oxygen; it does not really matter what species these plants are. What matters is that there are coastal forests to break storms and provide shelter and sustenance for fish larvae; it matters less whether there are more mangrove palms than buttonwoods.

A focus on nature's services begs the questions a service to whom or what? Plants can live without animals but animals cannot live without plants? And at what spatial or temporal scale should services be maintained? 

A third group of definitions of strong sustainability centres on ecosystem stability and resilience. This does not solve the fundamental problem of lack of specificity, because stability has to be defined in terms of either stocks or services.

The key feature of strong sustainability is that it imposes stronger constraints on human behaviour than does weak sustainability. Weak sustainability argues that all is fine as long as humans are fine. This is a utilitarian perspective. Strong sustainability wants other species to be fine too, even if it comes at the expense of humans. This is a naturalist perspective.

\section{A social construct}
The academic quest for a more precise definition of sustainable development than Brundtland's has thus led to a variety of definitions that are not particularly precise either. There is a third group of definitions: Sustainability is what we decide it to be.

This is a truism. Sustainability is not some property of the real world that we try to uncover. Sustainability is an abstract human desire, a social construct. But the people who argue this have a deeper motive. Brundtland's recommendation that development should be sustainable was so popular that it crowded out other objectives of government. Advocates of other worthy goals can either resist the new kid on the block, or seek to co-opt the item newly at the top of the agenda. People who argue that sustainability is a social construct also argue that it matters how society construes sustainability, a libertarian focus on procedure over outcome, and typically advocate deliberative, participatory democracy.

Sustainable development was initially about environmental quality, but this was quickly replaced by the three pillars of sustainable development: Environmental quality, distributional justice, and economic efficiency. The United Nations now has no fewer than 17 \emph{sustainable development goals}: a half one about economic growth, one about participatory democracy, three about the environment, and eleven and a half about development.\footnote{Sustainable development goal \#12 says that production and consumption should be sustainable. This is a tautology.} Sustainable development is now more development than sustainable.

Blurring concepts and re-purposing slogans is excellent politics but poor policy. Putting its three pillars under one heading masks the real trade-offs in sustainable development. Agriculture puts a lot of pressure on the environment. Cleaner forms of food production are more expensive. Greater environmental quality implies lower economic growth. Food is a necessary good. Poorer people spend a larger share of their income on food. Greater environmental quality implies a more unequal income distribution. Lumping everything under sustainable development hides these trade-offs. Vague words that appeal to every constituency is great politics. Ignoring the negative consequences of interventions is poor policy.

Jan Tinbergen showed that if you have $N$ policy problems, you need $N$ policy instruments. This is a corollary of Joseph-Louis LaGrange's work on constrained optimization. A simple example illustrate this. If you have two cars, you need two steering wheels. You can imagine a single car with two steering wheels, but negotiating a bend in the road requires a lot of coordination between the two drivers. Two cars that share a single steering wheel do fine as long as they go straight. Turning a corner is rather tricky. Killing two birds with one stone is so remarkable we made it a proverb.

Returning to example of food production, there are three objectives. A reduction in environmental pollution requires a change in farm practices. Economic efficiency demands that you do so for the lowest possible cost. As shown below, that is best done through a tax on emissions. Distributional justice dictates that part of the tax revenue be used to raise the income of the poorest. Giving three problems the same name suggests there is only one problem, and that one intervention is enough to solve it.